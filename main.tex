\documentclass[a4paper,12pt]{article}
\usepackage[utf8]{inputenc}
\usepackage[top=2cm, bottom=2cm, left=3cm, right = 1.5cm]{geometry}
\usepackage[T2A]{fontenc}
\usepackage[english,russian]{babel}
\usepackage{amsmath,amsfonts,amstext,amssymb}
\linespread{1.3}
\usepackage{multicol}
\setlength{\columnsep}{1cm}
\usepackage{graphicx}
\usepackage{caption}
\usepackage{xfrac}
\usepackage{etoolbox}
%\patchcmd{\thebibliography}{\section*{\refname}}{}{}{}
%\captionsetup[table]{format = hang, labelformat = simple, labelsep=endash}

%\captionsetup[figure]{format = hang, labelformat = simple, labelsep=endash, name = Рисунок}

\begin{document}

\renewcommand{\contentsname}{ }

\setcounter{secnumdepth}{-1}

 %РЕФЕРАТ
\begin{center}
\subsection{\bf{РЕФЕРАТ}}
\end{center}

 Одна из проблем, с которой сталкиваются исследователи звездных скоплений при оценивании их массы по функции светимости, это наличие в скоплении неразрешенных двойных систем. Особенно важно учитывать это обстоятельство при исследовании рассеянных звездных скоплений, где доля двойных может составлять десятки процентов. Цель настоящей работы - оценить, насколько учет неразрешенных двойных систем изменяет массу скопления.
 
В этой работе мы использовали функции светимости рассеянных звездных скоплений Ic2714, NGC 1912, NGC 2099, NGC 6834 и NGC 7142, которые были получены методом звездных подсчетов в ходе работы над созданием однородного каталога структурных и динамических характеристик рассеянных звездных скоплений по данным каталога точечных источников 2МАSS.
\newpage
%КОНЕЦ РЕФЕРАТА


%ОБОЗНАЧЕНИЯ И СОКРАЩЕНИЯ
\begin{center}
\section{\bf{ОБОЗНАЧЕНИЯ И СОКРАЩЕНИЯ}}
\end{center}

\begin{tabular}{l}
РЗС -- рассеянное звездное скопление\\
ЗС -- звездное скопление\\
ФС -- функция светимости скопления\\
$\alpha$ -- доля двойных звезд\\
$M_1$ -- масса главной компоненты двойной\\
$M_2$ -- масса вторичного компонента двойной\\
$q$ -- отношение масс компонент $M_2$ к $M_1$\\
$N_b$ -- число двойных звезд в определенном интервале звездных величин\\ 
$M_b$ -- масса двойных звезд в скоплении\\
$M_s$ -- масса одиночных звезд в скоплении\\
$M_{clb}$ -- масса скопления с учетом двойных звезд\\
$M_{cls}$ -- масса скопления, состоящего только из одиночных звезд\\
\end{tabular}

\newpage
%КОНЕЦ ОБОЗНАЧЕНИЯ И СОКРАЩЕНИЯ

\begin{center}
%ВВЕДЕНИЕ
\section{ВВЕДЕНИЕ}
\end{center}

Первые упоминания факта о наличии в звездных скоплениях большого количества неразрешенных двойных звезд можно обнаружить в работе Haffner \& Heckmann (1937) \cite{HH}.
С тех пор наличию двойных звезд было посвящено множество различных исследований. Например, Maeder в 1974 \cite{Maeder} показал, как располагается двойная на диаграмме звездная величина - показатель цвета в зависимости от отношения масс компонет $q=M_2/M_1$ (где $M_2$ это масса вторичной компоненты, а $M_1$ - масса главной компоненты двойной). В своей работе Hurley \& Tout (1998) \cite{HT} продемонстрировали, что последовательность, которая лежит выше главной последовательности звезд скопления на диаграмме CMD, образована двойными. Причем авторы этой работы выяснили, что коэффициент q имеет широкий разброс значений, то есть отношение масс компонент двойной может быть различным и может быть не равно 1 (что соответствует случаю одинаковых масс). 

Доля двойных звезд в шаровых звездных скоплениях относительно мала и обычно не превышает десяти процентов \cite{Milone2012} за несколькими исключениями. Однако, в работе Li с соавторами \cite{Li17} были обнаружены три шаровых скопления с гораздо большей долей двойных $(0.6-0.8)$

Рассеянные звездные скопления имеют значения доли двойных $\alpha \geqslant 30\%$ \cite{Li17, Boni, Khalaj, Sarro, alPer}. Этот процент все же меньше чем доля двойных среди звезд в окрестностях Солнца \cite{Duq1}. Было также получено, что доля двойных растет с увеличением массы главного компонента. Это чаще всего связано с динамической эволюцией звездного скопления \cite{KOP,Dorval}.

Очень важной характеристикой является функция распределения отношения масс компонент двойной $q$. В настоящее время среди исследователей нет точного согласия по поводу формы этого распределения. Согласно работе Duquennoy \& Mayor за 1991 год \cite{Duq2} функция распределения имеет максимум ближе к случаю маломассивных вторичных компонент, то есть далеко от 1. И наоборот, есть работы (например у Fisher с соавторами \cite{Fisher}), где пик расположен около случая равных масс компонент. Такой же пик был  обнаружен Maxted и другими \cite{Maxted} для маломассивных спектральных двойных в молодых скоплениях около $\sigma Ori$ и $\lambda Ori$. В работе Reggiani \& Meyer \cite{RM} была получена универсальная форма для распределения параметра $q$ для звезд типа Солнца и для карликов класса М в поле: $dN/dq \sim q^{\beta}$ с $\beta=0.25\pm0.29$. Это распределение можно считать плоским внутри интервала ошибок. 

Milone \cite{Milone2012} с соавторами определил, что  в интервале $q\in[0.5,1.0]$ распределение параметра $q$ у шаровых скоплений примерно плоское, с некоторыми отклонениями для нескольких случаев. А в работе Kouwenhoven и др. \cite{Kouwenhoven} использовались два других распределения: степенная зависимость $dN/dq \sim q^{\beta}$ для $q\in[q_0,1]$ и различных $\beta$, и Гауссово $dN/dq \sim exp[-(q-\mu_q)^2/2\sigma_q]$ для $q\in(0,1]$ с $\mu_q=0.23$ and $\sigma_q=0.42$.

Распределение отношения масс компонент двойной является ключом к восстановлению характеристик первоначальной популяции двойных. Было проведено множество численных исследований в данном направлении \cite{Kroupa2011,Geller2013,PR}.

Geller  и его соавторы \cite{Geller2013} показали на модели N-тел скопления NGC188, что распределение орбитальных параметров короткопериодичных двойных типа Солнца (с $P<1000^d$) должно быть примерно неизменным даже в пределах миллиарда лет эволюции. Это значит, что современные наблюдения двойных даже в старых скоплениях могут дать необходимую информацию о первоначальной их популяции. Parker \& Reggiani \cite{PR} показали, что общая доля двойных уменьшается в ходе динамической эволюции, но форма распределения $q$ не изменяется.  

Наличие неразрешенных двойных звезд в звездных скоплениях искажает оценки массы ЗС. Это следует учитывать как в случае звездных подсчетов, так и при оценке динамической массы (по значению дисперсии скоростей). Если набор звезд, отобранных по  оценке дисперсии скоростей (через лучевые скорости) содержит спектроскопические двойные, то значение дисперсии скоростей будет завышено. В результате, это приведет к завышенному значению массы скопления.

В работах Bianchini с соавторами \cite{Bianchini} и Kouwenhoven \& de Grijs \cite{KdeG} был сделан вывод, что учет населения двойных при оценке массы скопления очень важен. Селезнев с соавторами \cite{4337} отметили возможное влияние неразрешенных двойных на дисперсию скоростей, а затем и на оценку массы скопления NGC 4337

Когда масса скопления определяется через функцию светимости, полученную методом звездных подсчетов, то оценка массы будет меньше, чем в реальности, если не учитывать наличие в ЗС неразрешенных двойных. Это может быть легко объяснено тем, что масса двойной звезды больше массы одиночной с той же звездной величиной из-за сильной зависимости светимости звезд от их массы ($(L/L_{\odot})\sim(M/M_{\odot})^4$ \cite{CO}). 
Если одиночная звезда имеет ту же звездную величину, что и двойная, то их светимости также равны $L_s=L_1+L_2$ (где $L_s$ -- светимость одиночной звезды, а  $L_1$ и $L_2$ - светимости главного и вторичного компонента двойной соответственно). Тогда $M_s^4=M_1^4+M_2^4$, а значит $(M_1+M_2)^4=M_1^4+M_2^4+4M_1M_2^3+6M_1^2M_2^2+4M_1^3M_2=M_s^4+4M_1M_2^3+6M_1^2M_2^2+4M_1^3M_2$.
 Так как все величины положительные, то $(M_1+M_2)^4>M_s^4$, и $M_1+M_2>M_s$.

Наличие неразрешенных двойных было учтено в работе Khalaj \& Baumgardt (2013) \cite{Khalaj} для оценки массы РЗС Ясли. Авторы определили долю двойных в скоплении -- $30\pm5$ процентов, а затем нашли поправочный коэффициент, равный 1.35 (чтобы учесть неразрешенные двойные в скоплении, следует умножить массу скопления, полученную в предположении, что все звезды в нем одиночные, на данный множитель). 
К сожалению, Khalaj \& Baumgardt \cite{Khalaj} не объяснили, каким образом был получено значение этого множителя. 

Сейчас мы планируем новый проект, который будет включать в себя оценки масс достаточно большого количества РЗС при помощи функции светимости. Тем самым, интересно выполнить оценку того, как корректирующий множитель зависит от доли двойных $\alpha$ и распределения $q$ в довольно большом интервале значений этих параметров. Это и является целью настоящей работы.

Она организована следующим образом: сначала  описывается принятая модель и используемый алгоритм, затем приводится описание и анализ результатов для пяти РЗС NGC 1912, NGC 2099, NGC 6834, NGC 7142, и IC 2714. 
%КОНЕЦ ВВЕДЕНИЯ
 \newpage

%ОСНОВНАЯ ЧАСТЬ
\begin{center}
\section{ОСНОВНАЯ ЧАСТЬ}
\end{center}

Поправка к массе скопления будет зависеть от нескольких параметров.

Во-первых, необходимо учитывать распределение доли двойных по звездным величинам. Пока что в нашей модели используется равномерное распределение, то есть доля двойных звезд одинакова для всех звездных величин.

Во-вторых, отношение масс компонент двойной системы может быть различным. 
В этой работы были использованы плоское и два гауссовых распределения, которые изображены на рисунке (\ref{Funcs}).

\begin{figure}[h!]\centering
\includegraphics[width=7cm]{functionsforMK.png}
\caption{Распределения параметра q}
\label{Funcs}
\end{figure}

Для того, чтобы сказать точнее, какое распределение параметра q подходит для РЗС, следует построить модель и проверить ее на реальных скоплениях, у которых наблюдается последовательность двойных (например Плеяды, Ясли). Такая работа  планируется в дальнейшем.

Ранее уже рассматривали решения задач, связанных с массами двойных систем. Обзор таких работ был сделан Kouwenhoven \cite{Kouwenhoven} с соавторами и включает в себя следующие варианты постановки задач:

а)	Случайное моделирование пары. Массы обеих компонент случайным образом выбирается из распределения масс. 

б)	Случайное моделирование пары, ограниченной главным компонентом. Масса главного компонента выбирается из распределения масс, затем выбирается масса звезды-спутника с требованием, чтобы она была меньше

в)	Моделирование пары, ограниченный главным компонентом. Масса главного компонента выбирается из распределения масс, затем вычисляется масса звезды-спутника из распределения параметра q
 
г)	Расщепление ядра. Общая масса двойной системы берется из распределения масс. Масса главного компонента и звезды-спутника вычисляются по формулам связи массы системы и параметра q.

Наша задача отличается от решенных ранее, так как нам дана изначально светимость двойной системы, а не их суммарная масса. Нам нужно провести моделирование пары, ограниченной общей светимостью.

Для того, чтобы связать массу и светимость звезд скопления, мы используем квадратичную функцию масса-светимость, найденную Eker и соавторами в работе \cite{Eker}.
\begin{equation}
\begin{array}{l}
\log{L} = -0.705 (\log{M})^2  + 4.655 (\log {M}) - 0.025\\
 \qquad \qquad \quad \pm 41 \qquad \qquad \quad \pm42 \qquad \quad  \quad \pm10
\label{EkersEq}
\end{array},
\end{equation}
 где $L$ -- это светимость, выраженная в единицах светимости Солнца, а $M$  -- масса звезды, выраженная в массе Солнца.
 В своей работе Eker \cite{Eker} с соавторами получили еще и другие виды зависимостей, а также привели ряд аргументов, показывающих преимущество использования именно квадратичной формы.
 
 Мы допускаем, что в рассеянных звездных скоплениях возможна различная доля двойных. Поэтому мы исследуем различные варианты значений доли двойных в пределах 10-90\% , одинаковой для всех звездных величин звезд скопления
 %%%%%%%%%%%%%%%%%%%%%%%%%%%%%%%%%%%%%%%%%%%%%%%%%%%%%%%%%%%%%%%%
\subsection{1. Алгоритм}
При выполнении нашей работы мы используем функцию светимости для каждого скопления. Они были построены методом звездных подсчетов при помощи метода функции ядра в ходе работы над однородным каталогом структурных и динамических характеристик РЗС.%нужна ссылка на нашу работу?

Ниже в таблице ~\ref{clusters} отображены пределы видимых звездных величин для ФС в инфракрасной полосе J. Они определялись следующим образом: нижняя граница соответствует началу положительного участка значений ФС (отрицательные значения обуславливались тем, что плотность фона превышает плотность скопления при данной звездной величине), а верхняя граница определяется полнотой каталога 2MASS, по данным которого строились ФС.

\begin{table}[h!]
\caption{Параметры функций светимости скоплений}
\begin{tabular}{|c|c|c|c|c|}
\hline
Скопление & Начальная видимая J& Конечная видимая J& $(m-M)_0$&E(B-V)\\
\hline
IC2714 & 10.5 & 16.0 & 10.484 & 0.340\\
NGC1912 & 9.8 & 16.0 & 10.294 & 0.253\\
NGC2099 & 10.5 & 16.0 & 10.740 & 0.301\\
NGC6834 & 11.0 & 15.9 & 11.588 & 0.706\\
NGC7142 & 11.2 & 16.0 & 10.2 & 0.25\\
\hline
\end{tabular}
\label{clusters}
\end{table}

Мы разбиваем функцию светимости скопления на одинаковые интервалы звездных величин $\Delta $J (смотрите рисунок \ref{LF}). 

\begin{figure}[h!]\centering
\includegraphics[width=8cm]{LFunction_dJ.png}
\caption{Разбиение функции светимости скопления NGC 2099 на интервалы звездных величин}
\label{LF}
\end{figure}

Для каждого интервала считаем число звезд при помощи уравнения \eqref{Nstars}.
\begin{equation}
N= \Big[\int\limits_J^{J+\Delta J}{F(J)dJ}\Big] =\Big[<F(J)> \Delta J\Big]  
\label{Nstars}
\end{equation}

Значение $<F(J)>$ соответствует значению функции $F(J)$ на середине интервала $\Delta $J, которое находилось при помощи линейного интерполирования. Затем определяется число двойных, учитывая заданную долю двойных $\alpha$ по формуле \eqref{Nbinaries}. 


\begin{equation}
N_{b}=\Big[\alpha \int\limits_J^{J+\Delta J}{F(J)dJ}\Big]=\Big[\alpha <F(J)> \Delta J\Big]  
\label{Nbinaries}
\end{equation}

Значения величин $N$ и $N_{b}$ мы округляем до целых в силу их физического смысла. Из-за этого появляется ограничение на размер интервала $\Delta $J. При использовании линейного интерполирования интервал $\Delta $J не может быть слишком большим, иначе потеряется точность вычислений. 
При этом если мы будем брать маленькие интервалы, звездных величин, в которых число звезд $N$ не превышает 1, то при округлении будут получаться нулевые значения, которые, очевидно, вносят ошибку.

Из этих рассуждений было выбрано значение $\Delta $J, которое приблизительно равно $0.2^m$. При этом использование фиксированного $\Delta $J не удобно, так как не всегда интервалы звездных величин ФС могут быть нацело разделены на размер $\Delta $J, что приводит к наличию неучтенных промежутков звездных величин на краю ФС. 
Так как интервалы рассматриваемых звездных величин по ФС для разных скоплений имеют совсем незначительный разброс (смотрите табл.~\ref{clusters}), то мы фиксировали количество интервалов $n=30$, на которые разбивали ФС.  

Также мы считаем, что видимая звездная величина всех звезд из промежутка $\Delta $J равна звездной величине, соответствующей середине интервала. Зная видимую звездную величину, а также модуль расстояния и значение поглощения, по формуле \eqref{AbsMag} можем определить абсолютную звездную величину. Значение $A_J$ можно определить, пользуясь соотношением \eqref{Clr} 

\begin{equation}
M_J=J-(J-M_J)_0-A_J
\label{AbsMag}
\end{equation}

\begin{equation}
A_J=2.43 E(J-H)=0.9 E(B-V)
\label{Clr}
\end{equation}

Пользуясь таблицей изохроны для скопления, можно сопоставить абсолютной звездной величине звезды ее массу. Подставляя полученное значение массы в уравнение \eqref{EkersEq}, мы находим светимость звезды, которая соответствует данной массе, а значит и звездной величине.

Для каждой двойной системы из выбранного интервала звездных величин определяется свое отношение масс компонент двойной (параметр q). Затем строилась следующая система уравнений для двойной звезды:

\begin{equation}
\left\{
\begin{aligned}
&L=L_1+L_2\\
&\log{L_1}= -0.705(\log{M_1})^2  + 4.655(\log {M_1}) - 0.025\\
&\log{L_2}= -0.705(\log{M_2})^2  + 4.655(\log {M_2}) - 0.025\\
&q=\frac{M_2}{M_1}\\
\end{aligned}
\right.\label{MAINsystem}
\end{equation}

где L это светимость двойной (эта величина нам дана),
$L_1$ и $L_2 $ -- светимости компонент двойной звезды,
$M_1$ and $M_2$ -- массы звезд-компонент.
Значения модуля расстояния и поглощения цвета были взяты из каталога РЗС Локтина и Поповой \cite{LoPo}.

Параметр q, как было сказано ранее, выбирается для каждой пары. Эта процедура реализуется с помощью метода Неймана. В программу помещается файл с нужным распределением, которое задано с помощью таблицы. Затем набрасывается произвольным образом некоторое количество точек до тех пор, пока не найдется такая, которая лежит <<под>> графиком распределения. Абсцисса такой точки будет являться значением параметра q для данной пары. При плоском распределении будет подходить любая набрасываемая точка, а при гауссовых распределениях будут более вероятны точки с такими q, при которых наблюдается больший спектр ординат (что по сути соответствует максимумам распределений).

Введем следующие обозначения: заменим $\log{M_1}$ на $x$, как величину, которую требуется найти. Также для краткости обозначим коэффициенты буквенными символами: $a=-0.705 ,b=4.655, c=- 0.025$.

После некоторых преобразований мы можем вывести формулу, в которой будет выражена суммарная светимость двойной звезды:

\begin{equation}
\ln{L} = \ln{10} (ax^2+bx+c)+\ln{(1 + e^{\ln{10}(a(\log {q})^2+\log {q}(b+2ax))})}
\end{equation}

Нашей целью является определить $x$, поэтому мы строим функцию $f(x)$, которая обращается в ноль при таком $x$, который является решением системы (\ref{MAINsystem}):

\begin{equation}
f(x)=\ln{10} (ax^2+bx+c)+\ln{(1 + e^{\ln{10}(a(\log {q})^2+\log {q}(b+2ax))})} - \ln{L}
\end{equation}

Чтобы решить данное уравнение, мы используем метод Ньютона. Для этого мы вычисляем значение $x_{k+1}=x_k-\sfrac{f(x_k)}{f'(x_k)}$ до тех пор, пока разница $|x_{k+1} - x_k|$ не примет требуемую точность (где $f'(x_k)$ -- производная функции $f(x)$ в точке $x_k$). 

Метод Ньютона имеет ограничения в использовании. Для того, чтобы процесс итераций сошелся, должны быть выполнены некоторые условия.
Во-первых, мы должны правильно выбрать начальную точку, которая будет находиться близко к корню уравнения. Для этого мы строим сначала цикл с маленьким шагом в интервалах значения искомой массы $[0.08 \mathfrak{M}_{\odot};10 \mathfrak{M}_{\odot}]$ (из физических соображений значения для главной компоненты двойной системы), то есть для $x \in [-1.1,1.0]$. Цикл заканчивает свою работу, когда находит такое значение $x_i$, для которого $f(x_i)\cdot f(x_{i+1}) <0$. Это значит, что корень уравнения находится в интервале $x \in [x_i,x_{i+1}]$. Таким образом, мы считаем, что $x_i$ это и есть стартовая точка метода Ньютона.

Во-вторых, функция $f(x)$ должна быть непрервыной в области поиска нашего корня. Этот факт легко доказывается, так как $f(x)$ является комбинаций непрерывных на данном интервале $x$ функций, а именно: экспоненциальной, логарифмической, квадратичной.

Полученное в ходе решения уравнения методом Ньютона $x_k$ будет корнем уравнения, следовательно масса $M_1=10^x$ будет искомой массой главной компоненты двойной из системы уравнений (\ref{MAINsystem}).

Затем мы можем восстановить массу второй компоненты двойной по известному коэффициенту q, а затем и полную массу системы $M_1+M_2$.

Мы должны повторить данную процедуру для всех $N_{b}$  звезд и определить массу двойных в интервале звездных величин $\Delta J$, а затем просуммировать эти значения для всех интервалов, чтобы узнать массу двойных в скоплении.

После этого, чтобы получить массу скопления, нам нужно определить массу оставшихся одиночных звезд в количестве $N-N_{b}$ в каждом интервале $\Delta J$. Для определения массы звезды мы пользуемся таблицей изохроны, в которой приведены значения масс в соответствии с абсолютной звездной величиной. Отсюда и получаем массу всего скопления с учетом двойных звезд:

\begin{equation}
M_{clb} = M_{b}+M{s}
\end{equation}

Для определения разброса данный алгоритм был проведен 30 раз, чтобы получить некоторый набор значений масс. Конечным результатом стали среднее значение и величина дисперсии. Источником разброса значений масс скопления является статистическое определение параметра q из заданного распределения. При фиксированном $q=1$ случайная ошибка равна 0.

Чтобы сравнить массу скоплений с учетом двойных $M_{clb}$ с массой скопления, состоящего только из одиночных звезд, $M_{cls}$, необходимо получить $M_{s}$ из всех данных нам звезд. Это значит, нужно воспользоваться нашим алгоритмом со значением $\alpha = 0$.

\subsection{2. Результаты}

Мною были построены (Рисунки \ref{Plot_same}, \ref{Plot_flat}, \ref{Plot_G23}, \ref{Plot_G60}) графики зависимости отношения массы скопления с учетом двойных звезд к массе скопления, состоящего только из одиночных звезд ($\sfrac{M_{clb}}{M_{cls}}$) при разной доле двойных в скоплении $\alpha$. Бары ошибкок представлены (кроме случая равных компонент двойной) для скопления IC2714, так как ошибки имеют одинаковый порядок величины, а также чтобы избежать их наложения. 

Сначала для всех пяти скоплений был рассмотрен предельный случай равных компонент двойной (Рисунок \ref{Plot_same}). В таком случае поправочный коэффициент будет максимально отличаться от 1.

\vspace{1cm}
\begin{figure}[h!]\centering
\includegraphics[width=8.5cm]{alpha_plot_same.png}
\caption{Случай равных компонент двойной}
\label{Plot_same}
\end{figure}

Затем были построены графики для случая плоского распределения параметра $q$ (Рисунок \ref{Plot_flat}) и Гауссовых распределений $q$ с модой, равной 0.23 (Рисунок \ref{Plot_G23}) и 0.60 (Рисунок \ref{Plot_G60}).

\begin{figure}[h!]\centering
\includegraphics[width=8.5cm]{alpha_plot_flat.png}
\caption{Случай плоского распределения}
\label{Plot_flat}
\end{figure}

\vspace{2cm}

\begin{figure}[h!]\centering
\includegraphics[width=8.5cm]{alpha_plot_gauss.png}
\caption{Случай Гауссова распределения с $\mu =0.23$}
\label{Plot_G23}
\end{figure}
\vspace{3cm}

\begin{figure}[h!]\centering
\includegraphics[width=8.5cm]{alpha_plot_gr.png}
\caption{Случай Гауссова распределения с $\mu =0.60$}
\label{Plot_G60}
\end{figure}

Также были построены графики в тех же осях для каждого скопления в отдельности (Рисунки \ref{2714}, \ref{1912}, \ref{2099}, \ref{6834}, \ref{7142}). Это позволяет сравнить зависимость поправочного коэффициента от доли двойных при разных видах распределения параметра $q$. Откладываемые бары ошибкок представлены для плоского распределения по причине, указанной выше.

\begin{figure}[h!]\centering
\includegraphics[width=8.5cm]{plot_IC2714.PNG}
\caption{Зависимость поправочного коэффициента от доли двойных звезд для скопления IC2714}
\label{2714}
\end{figure}

\begin{figure}[h!]\centering
\includegraphics[width=8.5cm]{plot_NGC1912.PNG}
\caption{Зависимость поправочного коэффициента от доли двойных звезд для скопления NGC1912}
\label{1912}
\end{figure}
\vspace{3cm}

\begin{figure}[h!]\centering
\includegraphics[width=8.5cm]{plot_NGC2099.PNG}
\caption{Зависимость поправочного коэффициента от доли двойных звезд для скопления NGC2099}
\label{2099}
\end{figure}
\vspace{3cm}

\begin{figure}[h!]\centering
\includegraphics[width=8.5cm]{plot_NGC6834.PNG}
\caption{Зависимость поправочного коэффициента от доли двойных звезд для скопления NGC6834}
\label{6834}
\end{figure}

\begin{figure}[h!]\centering
\includegraphics[width=8.5cm]{plot_NGC7142.PNG}
\caption{Зависимость поправочного коэффициента от доли двойных звезд для скопления NGC7142}
\label{7142}
\end{figure}


\newpage




%КОНЕЦ ОСНОВНОЙ ЧАСТИ

%ЗАКЛЮЧЕНИЕ
\begin{center}
\section{\bf{ЗАКЛЮЧЕНИЕ}}
\end{center}
Мы исследовали, как масса скопления изменяется при учете наличия неразрешенных двойных звезд. При этом были сделаны следующие выводы:

а) Полученные зависимости можно с хорошей точностью назвать линейными.

б) Зависимости поправочного коэффициента для разных скоплений при использовании заданного распределения очень мало различаются в пределах ошибок, а значит можно использовать универсальную линейную функцию без зависимости от ФС.

в) Для каждого скопления поправочные коэффициенты для гауссовых распределений лежат внутри интервала ошибок плоского распределения. Из этого следует, что аппроксимация плоским распределением, которая используется в многих научных работах, обоснована. 

г) В работе Khalaj \& Baumgardt \cite{Khalaj} был приведен поправочный коэффициент 1.35 при доле двойных 30 процентов, но даже при предельном случае равных компонент двойной это значение не превышает 1.25. Так как авторами данной работы не был описан процесс нахождения поправочного коэффициента, то нельзя точно объяснить причины данных расхождений.

В дальнейшем планируется усложнять модель распределения доли двойных по звездным величинам, а также нахождение решения для систем с кратностью три и более. Кроме того, среди рассматриваемых задач стоит выяснение распределения отношения компонент двойной для рассеянных звездных скоплений.
\newpage
%КОНЕЦ ЗАКЛЮЧЕНИЯ

%СПИСОК ИСПОЛЬЗУЕМЫХ ИСТОЧНИКОВ
\begin{center}
\section{\bf{СПИСОК ИСПОЛЬЗУЕМЫХ ИСТОЧНИКОВ}}
\end{center}

\bibliography{Biblyo.bib}

%КОНЕЦ СПИСОК ИСПОЛЬЗУЕМЫХ ИСТОЧНИКОВ

\end{document} 
